\documentclass[12pt]{article}
% \usepackage[letterpaper, margin=0.65in]{geometry}

\setlength{\hoffset}{-0.75in}
\setlength{\voffset}{-1in}
\setlength{\textwidth}{7in}%{426pt}
\setlength{\textheight}{674pt}
\renewcommand{\baselinestretch}{1.2}
\setlength{\parskip}{0.05in}

\usepackage[utf8]{inputenc}
\usepackage[T2A]{fontenc}

\usepackage{indentfirst,amsmath,amssymb}
\usepackage{perpage}
\usepackage{footnote}
\MakePerPage{footnote}
\renewcommand{\thefootnote}{\fnsymbol{footnote}}
\renewcommand{\thefootnote}{\ifcase\value{footnote}
\or*\or**\or***\or\#\or\#\#\or\#\#\#\or$\infty$\fi}

\usepackage{xcolor}
\definecolor{link}{HTML}{820E0E}

\usepackage{minted}

% \usepackage[
%   backend=biber,
%   sorting=none,
%   % backref=true,
%   giveninits=true,
%   style=numeric-comp
% ]{biblatex}
% \addbibresource{refs.bib}

\title{MCFM ntuple patch}
\author{Ivan Pogrebnyak, MSU}
\date{\today}

\usepackage[
  hyperfootnotes=false,
  linktoc=all,
  colorlinks=true,
  linkcolor=link,
  citecolor=link,
  urlcolor=blue,
  pdfauthor={Ivan Pogrebnyak},
  pdftitle={MCFM ntuple patch},
  pdfkeywords={MCFM, Monte Carlo}
]{hyperref}

\begin{document}
\maketitle
\sloppy % no bleading into the margin

%%%%%%%%%%%%%%%%%%%%%%%%%%%%%%%%%%%%%%%%%%%%%%%%%%%%%%%%%%%%%%%%%%%%%
\section{Introduction}

The ability to save events, produced using a Monte Carlo generator, provides
much greater flexibility in comparison to saving specific histograms internally
populated by the events.
Saving ntuples of generated events allows to adjust cuts, add new histograms,
or change binning of already defined histograms without the need to rerun the
event generator.
Ntuples also allow users to utilize analysis software independent from the MC
generator program.
Having an intermediate format for event output allows for
analysis code to be written in any desired programming language.
Consequently, more analysis code, in whole or in part, may be reused from
previous studies.

Unfortunately, \texttt{MCFM} has been difficult to setup for ntuple output, and
the newer versions\footnote{
  \href{http://dx.doi.org/10.1007/jhep12(2019)034}{10.1007/JHEP12(2019)034},
  arXiv: \href{https://arxiv.org/abs/1909.09117}{1909.09117}.
} saw the feature removed entirely.
Here, I present an easy patch, that can be applied to any \texttt{MCFM}
version, to allow outputting ntuples.
As it is listed here, the code allows to write ntuples in
\texttt{ROOT}\footnote{
  \href{https://doi.org/10.1016/j.cpc.2009.08.005}{10.1016/j.cpc.2009.08.005},
  \url{http://root.cern.ch/}.
} format.
But any other format can be added with a few lines of code, as the patch
provides a generic API.

The main feature of the patch is the decoupling of the code that formats and
writes the ntuples from its interface.
This is achieved by compiling the writing code into a shared library,
containing functions with specific signatures, which comprise the API.
The shared library is dynamically loaded using \texttt{libdl},
the standard Linux dynamic linking library.
The shared library can be independently recompiled, if changes need to be made
to the ntuple writing code or its dependencies, such as the installed version
of \texttt{ROOT}, have changed.
The output format can also be changed without recompiling \texttt{MCFM}.

\clearpage
%%%%%%%%%%%%%%%%%%%%%%%%%%%%%%%%%%%%%%%%%%%%%%%%%%%%%%%%%%%%%%%%%%%%%
\section{Code structure}

The API functions have the following equivalent \texttt{C} declarations:
\begin{minted}{c}
  void* open_ntuple(const char* file_name)
  void close_ntuple(void* ntuple)
  void fill_ntuple(void* ntuple)
  void add_ntuple_branch_double(void* ntuple, const char* name, double* ptr)
  void add_ntuple_branch_float(void* ntuple, const char* name, float* ptr)
  void add_ntuple_branch_int(void* ntuple, const char* name, int* ptr)
\end{minted}

The \mintinline{c}{void* ntuple} is returned by \mintinline{c}{open_ntuple()}
and needs to be passed to the other functions.
It is a pointer to some implementation specific structure that represents
the ntuple writer.

\mintinline{c}{add_ntuple_branch_.*()} functions are used to pass pointers to
variables that will contain the values of ntuple branches.
The program using the interface needs to allocate these variables and overwrite
them for every event.
Once the values of all the variables are set, \mintinline{c}{fill_ntuple()}
needs to be called to add the event comprised of these values to the ntuple.
This is similar to how trees are written in \texttt{ROOT}.

Once all the events have been written, \mintinline{c}{close_ntuple()} should be
called to finish writing the ntuple file and to free the memory associated with
the writer structure.

For ntuple output in \texttt{ROOT} format, these functions are defined in the
\texttt{root\_ntuples.cc} file, written in \texttt{C++}.
This part cannot be written in \texttt{Fortran}, because \texttt{ROOT} does not
provide \texttt{Fortran} bindings.
Essentially, the reason for the particular structure of this patch is to allow
this part to be written in an arbitrary language with arbitrary dependencies.
The implementation is abstracted by encapsulating it in a shared library.
In order to load the library and call the API functions from \texttt{Fortran},
the interface file, \texttt{ntuple\_interface\_dl.c}, written in \texttt{C},
needs to be compiled with \texttt{MCFM}.
This part could probably be written in \texttt{Fortran}, but that would not
add any tangible benefit, and it is easier to do in \texttt{C}.
Finally, the \texttt{MCFM} specific \texttt{Fortran} interface is provided in
the file \texttt{ntuple\_interface.f}, which is a drop-in replacement for
\texttt{src/User/mcfm\_froot.f}.

%%%%%%%%%%%%%%%%%%%%%%%%%%%%%%%%%%%%%%%%%%%%%%%%%%%%%%%%%%%%%%%%%%%%%
\section{How to apply the patch}

Here are instructions on how to apply the patch to an existing \texttt{MCFM}
source code.
I use \texttt{MCFM-8.0} as the specific example, but the steps should be very
similar for all relatively recent versions.

\begin{enumerate}
\item Get the patch from \url{https://github.com/ivankp/mcfm_ntuple_patch}.
\begin{minted}{bash}
  cd /path/to/mcfm
  git clone https://github.com/ivankp/mcfm_ntuple_patch.git
\end{minted}

\item Compile the shared library.
\begin{minted}{bash}
  cd mcfm_ntuple_patch
  make
\end{minted}

\item Link (or copy) files to \texttt{src/User/}.
\begin{minted}{bash}
  cd ../src/User
  ln -s ../../ntuple/ntuple_interface_dl.c
  ln -s ../../ntuple/ntuple_interface.f
  cd ../..
\end{minted}

\item Edit the \texttt{MCFM} \texttt{makefile} to compile the added files.

Change the definition of the \texttt{USERFILES} variable:
\begin{minted}{make}
  # Check NTUPLES flag
  ifeq ($(NTUPLES),FROOT)
    # USERFILES += mcfm_froot.o froot.co # <-- comment this line out
    USERFILES += ntuple_interface_dl.o ntuple_interface.o # <-- add this line
\end{minted}

Add a rule for \texttt{.c} files:
\begin{minted}{make}
  %.o: %.c
        gcc -Wall -O3 -c $< -o $(OBJNAME)/$@
\end{minted}
It can be placed near line 2376, below the comment
\mintinline{make}{# for FROOT package}, but it doesn't really matter where.

\item Add an extra clause in \texttt{src/Need/mcfm\_exit.f} near the end.
\begin{minted}{fortran}
        call NTfinalize
    endif
  else ! add this line
    call NTfinalize ! add this line
  endif
\end{minted}

\item \texttt{MCFM} can now be recompiled and run.
\item
  In order to run with the new ntuple output, a link to (or copy of)
  the library compiled in step~2 needs to be present in the run directory,
  and has to be called \texttt{ntuples.so}.
\end{enumerate}

%%%%%%%%%%%%%%%%%%%%%%%%%%%%%%%%%%%%%%%%%%%%%%%%%%%%%%%%%%%%%%%%%%%%%
\section{Outlook}

If this patch is to be merged into a future release of \texttt{MCFM},
a few things need to be changed or added to fully incorporate it.
\begin{enumerate}
  \item The ntuples shared library can be made to compile together with
    \texttt{MCFM}, if \texttt{ROOT} is installed.

  \item Right now, the path to the ntuples shared library is hardcoded in
    \texttt{ntuple\_interface.f} to look for an \texttt{ntuples.so} file
    in the current directory, i.e. the directory where \texttt{MCFM} is run.
\begin{minted}{fortran}
  call load_ntuple_lib("./ntuples.so"//C_NULL_CHAR)
\end{minted}
    The current approach can be retained for its flexibility.
    But this requires a link to the library to be present in the run directory.
    This should be documented.

    Alternatively, the default path could be to some directory where
    the shared library will be put after it is compiled.

    An optional parameter can be added to the runcard to specify where to look
    for the library in case the user wants to use a custom one.
    This option can be added in either case.
    The option should specify the path to the \texttt{.so} file rather then to
    a directory, for greater flexibility.

  \item The \texttt{Fortran} interface file, \texttt{ntuple\_interface.f},
    can be cleaned up. I wrote it to imitate \texttt{src/User/mcfm\_froot.f},
    including dummy subroutines. This may not be necessary in \texttt{MCFM-9}.
\end{enumerate}

\clearpage
%%%%%%%%%%%%%%%%%%%%%%%%%%%%%%%%%%%%%%%%%%%%%%%%%%%%%%%%%%%%%%%%%%%%%
\section{Source code listings}

\subsection{\texttt{root\_ntuples.cc}}
\inputminted[
  fontsize=\footnotesize,baselinestretch=1,linenos,xleftmargin=\parindent
]{c++}{../root_ntuples.cc}

\subsection{\texttt{ntuple\_interface\_dl.c}}
\inputminted[
  fontsize=\footnotesize,baselinestretch=1,linenos,xleftmargin=\parindent
]{c}{../ntuple_interface_dl.c}

\subsection{\texttt{ntuple\_interface.f}}
\inputminted[
  fontsize=\footnotesize,baselinestretch=1,linenos,xleftmargin=\parindent
]{fortran}{../ntuple_interface.f}

% \printbibliography[title=References]
\end{document}

